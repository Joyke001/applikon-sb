\section{Tussenresultaten en kwaliteit}
Aan het einde van elke fase, genoemd in de projectactiviteiten hoofdstuk \ref{sec:activiteiten}, wordt er iets opgeleverd. Voor de eerste fase, Inlezen en onderzoek, is dit een document met alle bevindingen en een conclusie wat er gebruikt word voor het systeem. De ontwerp fase wordt afgerond met een de printplaat die besteld kan worden en een behuizing die kan worden geprint. Bij het afronden van de printplaat is er nog een tussen resultaat aanwezig, dit is de oplevering van het elektrisch schema.  In de volgende fase wordt er een werkende printplaat opgeleverd met een bijpassende behuizing. 

Na de hardware word de software geschreven voor de bioreactor en biocontroller module. Voordat er een werkende code wordt opgeleverd worden er tussenresultaten behaald. De tussenresultaten van de software zijn als volgt; meting starten en uitlezen, data formatteren, data draadloos versturen en uitlezen en data beveiliging/encription(mocht er genoeg tijd zijn).

Naast de software voor de systemen moet er ook een test model worden geschreven waar de metingen van een bedrade en draadloze sensor worden vergeleken. Het eerste tussenresultaat van het model is het uitlezen en verwerken van de data. Hier moet bepaald worden waaraan de betrouwbaarheid van het systeem afhankelijk is en hoe je dit in kaart kan brengen. Het eind resultaat is een model waar uit te lezen is wat de sensoren versturen hoeveel pakketten er worden verstuurd en hoe veel fouten er in de verstuurde pakketten zitten. 

Het project wordt afgerond met een conclusie of het systeem werkt en een model die aangeeft hoe betrouwbaar het draadloze sensor systeem is en aanbevelingen voor een vervolg product.

Om ervoor te zorgen dat de opgeleverde producten van goede kwaliteit zijn zullen ze goedgekeurd moeten worden door een collega. De elektronica wordt gecheckt en goedgekeurd door Arnold Meijer, electrical engineer van Getinge Applikon en de software wordt gedaan door Valentijn Polder, software engineer van Getinge Applikon.  