\section{Achtergrond}
Getinge Applikon, gelegen in Delft, ontwikkelt en produceert bioreactorsystemen. Deze systemen worden gebruikt om op laboratorium- of productieschaal een cultuur van cellen of micro-organismen te kweken. Om de condities tijdens dit proces te regelen zijn veel sensoren en actuatoren nodig, met als gevolg dat een laboratorium bioreactor veelal uit een grote hoeveelheid kabels, slangen en stekkers bestaat. Om de gebruiksvriendelijkheid van de bioreactorsystemen te verhogen onderzoekt Applikon de mogelijkheid om draadloze systemen toe te passen. 

Deze opdracht wordt vanuit Getinge Applikon begeleid door Astrid Kaljouw. Astrid is MSc. Life Science and Technology (Biotechnologie) en is bij Applikon Project Leader Product Development. Daarnaast zullen verschillende engineers van de productontwikkelingsafdeling beschikbaar zijn voor vragen en ondersteuning. 







