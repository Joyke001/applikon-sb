\section{Signal filtering}

In het huidige ontwerp van de smartcable worden er rc-filters toegepast voor een frequency cutoff van 1.6Hz te maken.

Er zijn een aantal aspecten waar passieve filters aan tekortkomen: ze worden sterk beïnvloed door de parameters van het elektriciteitsnet. De systeemimpedantiewaarde en de hoofdfrequentie van de resonantiefrequentie veranderen vaak met de werkomstandigheden; de frequentieband van de harmonische filtering is ook smal en alleen de hoofdfrequentie kan worden uitgefilterd. Versterk vanwege de parallelle resonantie enkele harmonischen van de orde; coördinatie tussen filtering en blindvermogencompensatie en spanningsregeling is moeilijk; naarmate de stroom die door het filter vloeit stijgt, kan de apparatuur overbelast raken; groot gewicht en volume vereisen meer ruimte; slechte loopstabiliteit en andere tekortkomingen. Dat is de reden waarom actieve filters met betere algehele prestaties steeds meer toepassingen krijgen.


Er wordt onderzocht of een actief filter toegepast kan worden om het DO-signaal te filteren. 
Vergeleken met passieve filters hebben actieve filters veel voordelen: snelle respons, geluid controleerbare prestaties; adaptieve functie, dynamische tracering en compensatie voor hogere harmonischen van het systeem, hoge stabiliteit, onafhankelijkheid voor de systeemimpedantie-invloed, vermogen om het optreden van resonantie, flikkeronderdrukking en reactieve componenten voor het systeem te voorkomen.


