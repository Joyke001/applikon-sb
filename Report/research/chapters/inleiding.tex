\section*{Inleiding}
Door de ontwikkeling van netwerk en communicatie technologie wordt het ongemak van kabels belicht. Draadloze netwerk systemen worden steeds meer gebruikt voor zowel de huishoudelijke apparaten als voor de industri{\"e}le automatisering. Op dit moment word de near field draadloos communicatie technologie op grote schaal gebruikt, denk hierbij aan bijvoorbeeld; Bluetooth, Wifi en Infrarood communicatie. 

Getinge Applikon, gelegen in Delft, ontwikkelt en produceert bioreactorsystemen. Deze systemen worden gebruikt om op laboratorium- of productieschaal een cultuur van cellen of micro-organismen te kweken. Om de condities tijdens dit proces te regelen zijn veel sensoren en actuatoren nodig, met
als gevolg dat een laboratorium bioreactor veelal uit een grote hoeveelheid kabels, slangen en stekkers bestaat. Om de gebruiksvriendelijkheid van de bioreactorsystemen te verhogen onderzoekt Applikon de mogelijkheid om draadloze systemen toe te passen.

Dit verslag beschrijft het onderzoek voor het uitbreiden van een Dissolved Oxygen probe van Getinge Applikon met een draadloos netwerk systeem. Hierbij worden de huidige hardware onder de loep genomen om een draadloos systeem mogelijk te maken. 